\section{The Quantum Mechanics of Spin}
\label{sec: quantum mechanics of spin}
\subsection{Pauli Matrices}
\label{sec: pauli matrices}

We have the following relations 

\begin{subequations}
	\begin{empheq}[box=\fbox]{align}
		\begin{split}
			& [S_i,S_j ]=\varepsilon_{ijk} i \h S_k\\
			& S^2\K{s,m}=\h^2 s(s+1)\K{s,m}\\
			& S_z \K{s,m}=\h m \K{s,m}\\
			& S_{\pm}\K{s,m}=(S_x \pm i S_y)\K{s,m} =\h \sqrt{s(s+1)-m(m\pm1)} \K{s,m\pm 1}
		\end{split}
	\end{empheq}
\end{subequations}


We have two states for spin 1/2
\begin{align*}
	\alpha=\begin{pmatrix}
		1\\0
	\end{pmatrix}; \beta=\begin{pmatrix}
		0\\1
	\end{pmatrix}
\end{align*}

And we can know 
\begin{align*}
	S^2 \alpha=\frac{3}{4}\h^2 \alpha; S^2 \beta=\frac{3}{4} \h^2 \beta 
\end{align*}

Let $S^2=\begin{pmatrix}a &b\\c&d\end{pmatrix}$, then we can get 
\begin{align*}
	S^2=\frac{3}{4}\h^2	\begin{pmatrix}
		1&0\\0&1
	\end{pmatrix}
\end{align*}

Similarly, we can get 
\begin{align*}
	S_z=\frac{\h}{2}\begin{pmatrix}
		1&0\\0&-1
	\end{pmatrix}; S_+=\h \begin{pmatrix}
		0&1\\1&0
	\end{pmatrix}; S_-=\h \begin{pmatrix}
		0&0\\1&0
	\end{pmatrix}
\end{align*}

Or 
\begin{align*}
	S_z=\frac{\h}{2}\begin{pmatrix}
		1&0\\0&-1
	\end{pmatrix}; S_x=\frac{\hbar}{2}\begin{pmatrix}
		0&1\\1&0
	\end{pmatrix}; S_y=\frac{\hbar}{2}\begin{pmatrix}
		0&-i\\i&0
	\end{pmatrix};
\end{align*}

And we let $S=\frac{\hbar}{2}\sigma$, where $\sigma$ is what we call Pauli Matrix

\begin{align*}
\boxed{
	\sigma_z=\begin{pmatrix}
		1&0\\0&-1
	\end{pmatrix}; \sigma_x=\begin{pmatrix}
		0&1\\1&0
	\end{pmatrix}; \sigma_y=\begin{pmatrix}
		0&-i\\i&0
	\end{pmatrix};}
\end{align*}

Or to make it easer to remember, we get a generally form 
\begin{align*}
	\sigma_a=\begin{pmatrix}
		\delta_{a3}&\delta_{a1}-i\delta_{a2}\\
		\delta_{a1}+i\delta{a2}&-\delta_{a3}
	\end{pmatrix}
\end{align*}

where $\delta_{ai}$ is \tcb{Kronecker delta}. \\

After we get the Pauli matrices, we can correspondingly get their eigenstates which are:

\begin{gather*}
	\psi_{z+}=\begin{pmatrix}
		1\\0
	\end{pmatrix}; \psi_{z-}=\begin{pmatrix}
		0\\1
	\end{pmatrix}\\
	\psi_{x+}=\frac{1}{\sqrt{2}}\begin{pmatrix}
		1\\1
	\end{pmatrix}; \psi_{x-}=\frac{1}{\sqrt{2}}\begin{pmatrix}
		1\\-1
	\end{pmatrix}\\
	\psi_{y+}=\frac{1}{\sqrt{2}}\begin{pmatrix}
		1\\i
	\end{pmatrix};\psi_{y-}=\frac{1}{\sqrt{2}}\begin{pmatrix}
		1\\-i
	\end{pmatrix}
\end{gather*}


\subsection{The Pauli Equation }
\label{sec: pauli equation}

We can set the electron' wavefunction as 
\begin{align}
	[\psi(\pmb{x})]=\begin{bmatrix}
		\phi_1(\pmb{x})\\ \phi_2(\pmb{x})
	\end{bmatrix}	
\end{align}

Where $\pmb{x}=(x,y,z,t)$. Then we can recast the Schrodinger Equation 
\begin{align}
	\Big (\hat{H}-i\hbar \frac{\partial}{\partial t} [I]\Big)[\psi(\pmb{x})]=[0]
	\label{equ: pauli equation}
\end{align}

This is a set of two simultaneous differential equations for the two components of the spinor wavefunction--$\phi_1$ and $\phi_2$. And this is referred to as the \tcb{\ti{Pauli Equation}}.\\ 

Right now, we can calculate the expected value of the spin components with special expressions. 
\begin{gather}
	\BK{S_n}=\frac{\hbar}{2}[\psi(\pmb{r},t)]^\dagger \sigma_n [\psi(\pmb{r},t)]
\end{gather}


Specifically, 
\begin{gather}
	\BK{S_x}=\frac{\hbar}{2} \begin{bmatrix}
		\phi^\dagger_1(\pmb{x})& \phi^\dagger_2(\pmb{x})
	\end{bmatrix} \begin{bmatrix}
		0&1\\
		1&0
	\end{bmatrix} \begin{bmatrix}
		\phi_1(\pmb{x})\\ \phi_2(\pmb{x})\end{bmatrix}=\hbar Re(\phi_1^\dagger(\pmb{x})\phi_2(\pmb{x}))\\
		\BK{S_y}=\hbar Im(\phi^\dagger_1(\pmb{x}) \phi_2(\pmb{x}))\\
		\BK{S_z}=\hbar (|\phi_1(\pmb{x})|^2-|\phi_2(\pmb{x})|^2)
\end{gather}

\subsection{Dirac Equation}
\label{sec: dirac equation}

Well, Pauli's theory about spin is non-relativistic. The task is finished by Paul Dirac with his Relativistic Wave Equation. 

\subsubsection{Klein-Gordon Equation}
\label{sec: klein gordon equation}

As soon as  Schrodinger equation was proposed, the relativistic wave equation was also put forward. That is Klein-Gordon Equation
\begin{align*}
	-\hbar^2 \frac{\partial^2}{\partial t^2}\psi=(-\h^2c^2 \pmb{\nabla}^2+m^2c^4)\psi
\end{align*}

But this equation can not describe one-single particle, instead it can be used to describe a field--\tcb{scalar field}. The reason mainly comes from the second derivative.  Well let's see the details.\\

For non-relativistic case, we have equation
\begin{align*}
	i\hbar \frac{\partial}{\partial t}\psi=-\frac{\hbar^2}{2m}\pmb{\nabla}^2\psi
\end{align*}

And we can set
$$\rho=\psi^\star \psi$$
$$\pmb{j}=-\frac{i\hbar}{2m}(\psi^\star\pmb{\nabla}\psi-\psi \pmb{\nabla}\psi^\star )=\frac{\BK{p}}{m}$$

Then we can get one equation 
\begin{align*}
	\boxed{\frac{\partial}{\partial t}\rho+\pmb{\nabla\cdot j}=0}
\end{align*}

Now, let's prove this, first we have one equation
\begin{align*}
	-i\hbar \frac{\partial}{\partial t}\psi^\star=-\frac{\hbar^2}{2m}\pmb{\nabla}^2\psi^\star
\end{align*}

Then we take 
\begin{align*}
	i\hbar \frac{\partial}{\partial t}(\psi^\star \psi)&=i\hbar (\dot{\psi}^\star\psi+\psi^\star \dot{\psi})\\
	&=\frac{\hbar^2}{2m}\pmb{\nabla}^2 \psi^\star \psi-\frac{\hbar^2}{2m}\psi^\star \pmb{\nabla}^2\psi\\
	&=\frac{\hbar^2}{2m}\pmb{\nabla}\cdot(\pmb{\nabla}\psi^\star \psi-\psi^\star \pmb{\nabla}\psi)
\end{align*}

Finally, we can get 
\begin{gather*}
	\frac{\partial}{\partial t}(\psi^\star \psi)=-\frac{i \hbar}{2m}\pmb{\nabla}\cdot (\pmb{\nabla}\psi^\star \psi-\psi^\star \pmb{\nabla}\psi)\\
	\Downarrow\\
	\frac{\partial}{\partial t}\rho=-\pmb{\nabla}\cdot \pmb{j}
\end{gather*}

This is actually the conversation of probability. Well but this is not very true for Klein-Gordon Equation. With the same way, we can have 
\begin{align*}
	-\hbar^2 \frac{\partial}{\partial t}(\psi^\star \frac{\partial}{\partial t}\psi-\psi \frac{\partial}{\partial t}\psi^\star)=-\hbar^2 c^2 \pmb{\nabla}\cdot(\psi^\star \pmb{\nabla}\psi-\psi \pmb{\nabla}\psi^\star)
\end{align*}

Now we need to re-set the symbols
$$\rho=\frac{i\hbar}{2mc^2} (\psi^\star \frac{\partial}{\partial t}\psi-\psi \frac{\partial}{\partial t}\psi^\star)$$
$$\pmb{j}=-\frac{i\hbar}{2m}(\psi^\star \pmb{\nabla}\psi-\psi \pmb{\nabla}\psi^\star)

We can still get 
\begin{align*}
	\frac{\partial}{\partial	t}\rho+\pmb{\nabla}\cdot \pmb{j}=0
\end{align*}

\tcb{\ti{Well, we still get a equation looks like the conversation of probability equation. But here has some difference. What is the meaning of $\rho$ here, it can't be seen as probability density simply. And it's not always positive. Well this ``minuse probability" problem can't be solved when we treat Klein-Gordon Equation as suitable relativistic wave equation. Then Dirac mentioned an equation named after him.}}



\subsubsection{Dirac Equation}

Klein-Gordon Equation can be written as 
\begin{align*}
	\Big[ (i\h\frac{\partial}{c \partial t} )^2-\sum_1^3 (-i\hbar \frac{\partial}{\partial x_r} )-m_0^2c^2  \Big]\psi(x,y,z,t)=0
\end{align*}

When there is a field $\vec{A}=(A_0,A_x,A_y,A_z)$, the equation will become 
\begin{align*}
	\Big[ (i\h\frac{\partial}{c \partial t} +eA_0)^2-\sum_1^3 (-i\hbar \frac{\partial}{\partial x_r} +eA_r)-m_0^2c^2  \Big]\psi(x,y,z,t)=0
\end{align*}

Well, Dirac think the relativistic wave equation should also be first order differential respect to time. And then he thought the right form should be 
\begin{align*}
	\Big[ (i\h\frac{\partial}{c \partial t} +eA_0)-\sum_1^3 \alpha_r(-i\hbar \frac{\partial}{\partial x_r} +eA_r)-\alpha_0m_0^2c^2  \Big]\psi(x,y,z,t)=0
\end{align*}





